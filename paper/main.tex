\documentclass[12pt]{article}
\usepackage[margin=1in]{geometry}
\usepackage{hyperref}
\usepackage{setspace}
\usepackage{natbib}
\setlength{\parskip}{0.7em}
\setlength{\parindent}{0pt}
\doublespacing

\title{The Whole and the Slice: A Trinitarian, Christ-Patterned Ontology of Reality and Consciousness}
\author{Working Paper}
\date{\today}

\begin{document}
\maketitle

\begin{abstract}
This paper proposes a unified ontological and theological framework in which all finite lives are constraint-shaped appearances of a single undivided Whole. Each finite self is animated by attention and shaped by deep patterning --- non-narrative configurations accumulated through prior experience. The Whole is structured by a value gradient oriented toward coherence, life, and self-giving love. Evil is framed not as substance or force, but as destructive misalignment and weaponized fragmentation of this gradient. Within this model, lives function as learning trajectories through which alignment and misalignment are explored under pressure. The Christ pattern reveals the form of perfect alignment under constraint, and the Spirit sustains participation in this trajectory. A structural mapping of the Trinity --- as Whole-as-ground, Whole-in-form, and Whole-as-attention --- supports this metaphysical vision. The framework integrates insights from theology, process philosophy, cognitive science, and contemplative traditions, offering a coherent model of personal identity, religious life, and divine action.
\end{abstract}

\section{Introduction}

This paper presents a constructive ontology that brings together insights from contemporary systems science, process metaphysics, Christian trinitarian theology, and nondual traditions. Its central claim is that there is only the \emph{Whole} --- an undivided field of reality that has no outside and no ultimate division. What we ordinarily call individuals, selves, or subjects are not substances separate from the Whole, but \emph{slices}: the Whole appearing under specific limits.

This central intuition is consistent with the witness of Christian Scripture, which repeatedly affirms that all things originate within, subsist through, and return to a single divine reality. “For from Him and through Him and to Him are all things” (Romans 11:36). “In Him we live and move and have our being” (Acts 17:28). “He is before all things, and in Him all things hold together” (Colossians 1:17). The Gospel of John grounds this unity in the Logos: “All things came into being through Him… In Him was life, and that life was the light of all humankind” (John 1:3–4). These passages, taken together, express precisely the metaphysical claim of this framework: reality itself is a single divine Whole, and finite creatures are modes of its self-expression.

The framework is motivated by several pressures at once. Scientific and computational approaches to the world emphasize unified dynamics, information flow, and systemic regularities rather than isolated objects \citep{wolfram2002nks, friston2010free}. Religious and contemplative traditions insist on a dimension of meaning, value, and interiority that resists reduction to mechanisms. Christian theology adds the doctrine of the Trinity and the incarnation of Christ. Nondual traditions such as Advaita Vedanta articulate a vision of a single underlying reality manifesting as many \citep{vivekanandajnana, sarvapriyananda2020advaita}.

Scripture also recognizes these multiple pressures on human understanding. The Psalmist declares that “the heavens declare the glory of God; the skies proclaim the work of his hands” (Psalm 19:1), affirming a cosmic order intelligible through natural observation. At the same time, the prophets insist that true knowledge comes from divine disclosure: “Call to me and I will answer you and tell you great and unsearchable things you do not know” (Jeremiah 33:3). The New Testament unites these threads, teaching that Christ is both the rational structure of creation and the personal self-disclosure of God: “In the beginning was the Word… all things were made through Him” (John 1:1,3). Thus, Scripture situates empirical, contemplative, and revelatory forms of knowledge within one coherent reality—an approach mirrored in this framework’s synthesis of science, theology, and nondual insight.

This multivalent witness—scientific, contemplative, philosophical, and scriptural—reveals a coherence that Scripture itself affirms. Paul writes that “since the creation of the world God’s invisible qualities—his eternal power and divine nature—have been clearly seen, being understood from what has been made” (Romans 1:20), suggesting that the natural order points toward a unified reality. At the same time, Jesus reveals the experiential and relational dimension of this unity: “I am in the Father and the Father is in me” (John 14:10), and prays that humanity “may all be one, just as you, Father, are in me, and I in you” (John 17:21). These teachings point not only to metaphysical unity but to participatory union, grounding this framework’s move from a merely descriptive ontology to a relational, Christ-patterned account of the Whole and its slices.

Across these discourses runs a cluster of shared themes. Each in its own language speaks of a single dynamic ground of reality, finite centers of experience shaped by deep patterns and histories, and a directional pull toward richer forms of value. Process philosophy names this the creative advance of actual occasions ordered by divine aims \citep{whitehead1978process, whiteheadSEP}. Predictive processing and active inference describe agents that minimize surprise by updating internal models and reweighting attention \citep{friston2010free}. Vedantic authors speak of Brahman manifesting as many \emph{j\={\i}vas} limited by dispositions (\emph{sa\m sk\={a}ras}) and ignorance that can be transformed through knowledge and practice \citep{vivekanandajnana, sarvapriyananda2020advaita}. Contemporary computational metaphysics, especially work on the Ruliad and observer theory, treats observers as finite, computationally bounded structures that carve stable regularities out of an underlying sea of computation \citep{wolfram2002nks, wolfram2021ruliad, wolfram2023observer}. Christian Scripture and patristic theology insist that the one God creates through the Word, sustains all things, and draws creatures into a pattern of self-giving love revealed in Christ and enacted by the Spirit \citep{athanasius2011incarnation, augustine1991trinity, rahner1997trinity, wright1996jesus}.

From a biblical standpoint, this multi-perspectival approach is not only permissible but expected. Scripture consistently calls creation to be interpreted at the intersection of reason, revelation, and lived participation in divine life. “Test everything; hold fast to what is good” (1 Thessalonians 5:21) encourages philosophical discernment. “The Spirit will guide you into all truth” (John 16:13) grounds contemplative insight. “All Scripture is God-breathed and useful for teaching… and training in righteousness” (2 Timothy 3:16) names revelation as a normative anchor. And Christ Himself synthesizes these dimensions, declaring, “I am the way, and the truth, and the life” (John 14:6). This framework therefore approaches ontology—and the very structure of consciousness—through the same integrated lens that Scripture prescribes: a harmony of scientific clarity, contemplative depth, and Christ-centered revelation.

The present proposal offers a single coherent picture within which these concerns can be seen as different articulations of the same underlying structure. It begins with the Whole and its slices, introduces the notions of constraints, deep patterning, and attention, and then develops the idea of a value gradient toward coherence and love. On this basis it reframes evil, death, and the meaning of individual lives as learning trajectories of the Whole under constraint. Finally, it integrates these concepts into a Christ-patterned, trinitarian interpretation: the Whole as Father, the Whole in form as Son, and the Whole as attention as Spirit, and considers the implications for religion, agency, and debates about consciousness.

The argument proceeds by first defining the core vocabulary (Whole, slice, deep patterning, attention, value gradient), then unfolding a metaphysical picture of Whole and slices, evil, and lives as learning trajectories, developing the Christ pattern and a trinitarian mapping of Father, Son, and Spirit, and finally considering implications for religion and human agency and situating the proposal in comparative perspective before concluding.

\section{Core Concepts}

Before developing the framework in detail, it is useful to state its central terms in a compact way.

\textbf{Whole.} The Whole is undivided reality: all that is, including all the ways things can appear. There is no outside to the Whole. It is the ground and field of all processes, relations, and meanings.

Scripture affirms this ontological unity in numerous passages. ``The Lord is the everlasting God, the Creator of the ends of the earth'' (Isaiah 40:28) presents God as the ground of all that is. Paul declares that ``in Him we live and move and have our being'' (Acts 17:28), identifying creaturely existence as participation in divine reality. Colossians describes Christ---the visible form of the invisible God---as the one ``in whom all things hold together'' (Colossians 1:17). And in 1 Corinthians 15:28, the eschatological aim of creation is stated directly: ``God will be all in all.'' These texts articulate the same metaphysical insight as this framework: that reality is one undivided Whole, and all finite things are manifestations of its being.

\textbf{Slice.} A slice is the Whole appearing under a particular set of limits. A slice includes a body, a nervous system, a developmental history, a cultural and linguistic frame, and finite resources of time, energy, and attention. A self is the Whole-as-this-slice, not a separate substance alongside the Whole.

Scripture testifies to this participatory mode of existence. Human beings are said to bear ``the image and likeness of God'' (Genesis 1:26), suggesting that finite persons express divine reality under creaturely conditions. Paul teaches not merely imitation but participation: ``It is no longer I who live, but Christ lives in me'' (Galatians 2:20), and again, ``We have the mind of Christ'' (1 Corinthians 2:16). Jesus Himself articulates the paradox of finite-divine interpenetration when He quotes, ``You are gods'' (John 10:34; Psalm 82:6). These passages do not collapse the distinction between creature and Creator, but they affirm that the life of each person is grounded in, sustained by, and expressive of the divine Whole. A slice is therefore not an independent substance but the Whole-as-this-person, manifesting under the constraints of embodiment, history, and perspective.

\textbf{Constraints.} Constraints are the structural and historical limits that shape what a slice can sense, model, and do: biological architecture, environmental conditions, social and technological context, and inherited patterning. They define the interface through which the Whole shows up locally.

Scripture consistently affirms these creaturely limitations. ``For He knows our frame; He remembers that we are dust'' (Psalm 103:14) grounds human limitation in our created constitution. Paul describes our epistemic constraint explicitly: ``For now we see in a mirror dimly, but then face to face; now I know in part'' (1 Corinthians 13:12). Jesus highlights the limits of embodied existence when He tells His disciples, ``The spirit is willing, but the flesh is weak'' (Matthew 26:41). Even our temporal finitude is emphasized: ``All flesh is like grass\ldots{} the grass withers and the flower falls'' (1 Peter 1:24). These passages reveal that human beings encounter reality through the bounded interface of their embodiment, history, and perception---precisely the condition described here as constraint.

\textbf{Deep patterning.} Deep patterning is the largely non-verbal background structure of a slice: emotional imprints, learned reactions, attachment styles, value hierarchies, and implicit expectations of safety and meaning. It is like the ``weights'' of the slice, lived as dispositions to interpret and respond rather than as explicit memories.

Scripture frequently names this domain of deep, pre-reflective structure as the ``heart.'' ``The purposes of a person's heart are deep waters'' (Proverbs 20:5), indicating layers of disposition beneath conscious awareness. Jesus teaches that patterns embedded in the heart shape perception and action: ``A good person brings good things out of the good stored up in their heart\ldots{} for the mouth speaks what the heart is full of'' (Luke 6:45). The Old Testament often describes generational and formative patterning: ``The sins of the fathers'' shaping future dispositions (Exodus 20:5--6), and the Psalmist prays for transformation at this level: ``Create in me a clean heart, O God, and renew a right spirit within me'' (Psalm 51:10). Paul speaks of the ``old self,'' with its entrenched patterns, and the need to ``be renewed in the spirit of your mind'' (Ephesians 4:22--23). All these passages point to a biblical understanding of deep patterning as the accumulated, largely implicit structure that guides attention, interpretation, and behavior---precisely the domain this framework identifies as the non-narrative core of the slice.

\textbf{Attention.} Attention is the shifting focus of awareness within a slice. It determines what comes to the foreground, which options are noticed, and what feels salient or irrelevant. Over time, patterns of attention can reshape deep patterning, reinforcing some configurations and weakening others.

Scripture frames attention as a decisive spiritual and existential faculty. Jesus teaches, ``The eye is the lamp of the body; if your eye is single, your whole body will be full of light'' (Matthew 6:22), indicating that the focus of awareness determines the quality of one’s being. Paul emphasizes the formative power of attentional direction when he instructs believers to ``set your minds on things above, not on earthly things'' (Colossians 3:2) and contrasts ``the mind set on the flesh'' with ``the mind set on the Spirit,'' the latter being ``life and peace'' (Romans 8:5--6). The Psalmist expresses the attentional center of spiritual life: ``I have set the Lord always before me'' (Psalm 16:8), and again, ``One thing I seek\ldots{} to behold the beauty of the Lord'' (Psalm 27:4). Hebrews exhorts disciples to ``fix your eyes on Jesus'' (Hebrews 12:2), grounding attention explicitly in the Christ-pattern. These passages reveal that biblical spirituality understands attention as the primary interface through which perception, desire, and action align with---or deviate from---the divine Whole.

\textbf{Value gradient.} The Whole is not morally neutral. It has a value gradient, a built-in tilt toward patterns that sustain coherence, life, mutuality, and self-giving love, and away from patterns that collapse into fragmentation, cruelty, and meaning-destruction. This gradient is not an external rule but the way reality itself favors certain forms of order over others.

Scripture portrays this value-structure as intrinsic to both God's character and creation itself. ``God is love'' (1 John 4:8) establishes love as the fundamental nature of the Whole. Jesus identifies love as the organizing principle of moral and spiritual reality: ``On these two commandments hang all the Law and the Prophets'' (Matthew 22:40). Paul declares that ``love is the fulfillment of the law'' (Romans 13:10) and contrasts the destructive outcomes of the flesh with ``the fruit of the Spirit,'' which is love, joy, peace, patience, kindness, goodness, faithfulness, gentleness, and self-control (Galatians 5:22--23). The Psalms describe a cosmos structured by divine wisdom: ``Righteousness and justice are the foundation of your throne; love and faithfulness go before you'' (Psalm 89:14). In the Sermon on the Mount, Jesus depicts love as the telos of embodied life, commanding, ``Be perfect, therefore, as your heavenly Father is perfect'' (Matthew 5:48)---perfection here meaning wholehearted love. These passages together reveal a biblical value gradient: creation is not morally neutral but ordered toward the coherence, compassion, and relational integrity that reflect God's own being.

\textbf{Aim.} Aim is the deep direction of a slice's deep patterning and attention: toward truth or deception, love or control, coherence or fragmentation. Creativity at the slice level is structurally neutral; aim determines whether it cooperates with or resists the Whole's value gradient.

Scripture repeatedly emphasizes this inner orientation as the decisive factor in human life. ``As a man thinks in his heart, so is he'' (Proverbs 23:7), grounding identity in internal direction. Jesus teaches the same principle through path imagery: ``Enter through the narrow gate\ldots{} the way is hard that leads to life'' (Matthew 7:13--14). The Psalmist frames aim as the posture of the heart toward God: ``Teach me Your way, O Lord; I will walk in Your truth; unite my heart to fear Your name'' (Psalm 86:11). Paul describes the transformative reorientation of aim in cruciform terms: ``Those who belong to Christ Jesus have crucified the flesh with its passions and desires'' (Galatians 5:24). He presents aim as the deep alignment of desire with the Spirit: ``Walk by the Spirit, and you will not gratify the desires of the flesh'' (Galatians 5:16). These passages reveal that in biblical thought, the inner aim of the person determines the arc of their life, shaping whether they move toward coherence and love or toward fragmentation and self-enclosure.

\textbf{Alignment and misalignment.} Alignment is the state in which a slice's aim and deep patterning lean with the Whole's value gradient, expressing self-giving love, integrity, responsibility, and mercy. Misalignment is active movement against this gradient: domination, resentment, exploitation, and ``burn it down'' postures that amplify fragmentation.

Biblically, alignment corresponds to walking in the light, abiding in Christ, and living by the Spirit. ``If we walk in the light, as He is in the light, we have fellowship with one another'' (1 John 1:7), linking alignment with relational coherence. Jesus describes alignment as abiding: ``Abide in me, and I in you\ldots{} apart from me you can do nothing'' (John 15:4--5). Paul frames alignment as conformity to Christ's character: ``Let this mind be in you which was also in Christ Jesus'' (Philippians 2:5). Conversely, misalignment corresponds to sin, hardness of heart, and fragmentation. ``Everyone who sins is a slave to sin'' (John 8:34) depicts misalignment as a self-reinforcing bondage. Proverbs describes the dissociated state of misalignment as inner fracture: ``The way of the wicked is like deep darkness; they do not know over what they stumble'' (Proverbs 4:19). Paul contrasts the two trajectories starkly: ``The mind governed by the flesh is death, but the mind governed by the Spirit is life and peace'' (Romans 8:6). These passages show that alignment and misalignment are not abstract categories but lived orientations toward or against the divine Whole.

\textbf{Evil.} Evil is not a rival being but destructive misalignment with the value gradient. It shows up as purposeless cruelty, the shattering of trust, the corrosion of meaning, and the cultivation of fragmentation for its own sake. It can use both chaos and highly structured systems, as long as the result is to damage coherence and relationship.

Scripture consistently presents evil not as an independent force but as distortion, corruption, or privation of the good. Isaiah declares, ``Woe to those who call evil good and good evil'' (Isaiah 5:20), revealing evil as a reversal or misordering of value rather than a competing essence. John affirms that ``God is light; in Him there is no darkness at all'' (1 John 1:5), indicating that darkness has no substance of its own but is the absence of divine light. Paul describes evil as ``the works of the flesh'' (Galatians 5:19--21), a catalog of relational and moral fragmentations. Jesus teaches that evil arises from misaligned interior patterning: ``Out of the heart come evil thoughts'' (Matthew 15:19). The Old Testament frames evil as corruption: ``All flesh had corrupted their way on the earth'' (Genesis 6:12). And Athanasius echoes Scripture when he writes that sin tends toward non-being. Together these passages show that evil in the biblical witness is dis-integration---an inward collapse away from coherence and love---precisely the form of misalignment described in this framework.

\textbf{Learning trajectories.} Each finite life is a learning trajectory of the Whole under that slice's limits. As experiences unfold, deep patterning is updated; when the slice dissolves, the interface is gone but the deep patterning is retained in the Whole's overall configuration. In this sense, nothing lived is wasted; alignment and misalignment both leave lasting marks on reality's structure.

Scripture portrays human life as a divinely guided trajectory of formation, refinement, and transformation. Paul assures believers that ``He who began a good work in you will carry it on to completion'' (Philippians 1:6), framing life as a progressive divine craftsmanship. James teaches that trials shape maturity: ``The testing of your faith produces perseverance\ldots{} that you may be mature and complete'' (James 1:3--4). Hebrews depicts suffering and discipline as pedagogical: ``God disciplines us for our good, in order that we may share in His holiness'' (Hebrews 12:10). Paul describes inner transformation in developmental terms: ``We all\ldots{} are being transformed into His image with ever-increasing glory'' (2 Corinthians 3:18). Jesus uses the metaphor of a seed: ``Unless a grain of wheat falls into the ground and dies, it remains alone; but if it dies, it bears much fruit'' (John 12:24). Even the continuity of deep patterning across death resonates with Scripture: ``Nothing will be lost'' (John 6:39) and ``your labor in the Lord is not in vain'' (1 Corinthians 15:58). These texts support the framework's claim that each life is a learning trajectory of the Whole under constraint, where experiences---whether aligned or misaligned---contribute to the ongoing shaping of the person's deepest configuration.

\section{The Whole and the Slice}

The first axiom of the framework is that there is only the Whole. The Whole is undivided reality: all that is, including all possible ways of appearing. There is nothing outside it. Even what might be called ``nothingness'' is only the Whole appearing as absence or silence. In this respect the Whole shares features with the one processual reality of Whitehead, in which actual occasions are internally related and prehend the whole past \citep{whitehead1978process}, and with the nondual Brahman of Vedanta, which manifests as many through limiting conditions \citep{vivekanandajnana}.

The second axiom is that what we ordinarily call a self or an individual is the Whole appearing under constraints. A \emph{slice} is defined as the Whole under a specific configuration of limitations: a body, a nervous system, a developmental history, a cultural and linguistic frame, and finite resources of time and energy. A slice functions as an interface through which the Whole experiences, acts, and learns under these constraints.

On this view, the self is not a separate substance added to the universe. It is the Whole-as-this-person. The distinction between Whole and slice is not a division in being but a difference in mode. The Whole is capable of expressing itself in many such modes, each with its own vantage point and blind spots. This resonates with computational and observer-based metaphysics in which different observers are finite structures embedded in a single underlying computational universe, or ``Ruliad'', and extract different reduced representations of it \citep{wolfram2021ruliad}. It is also structurally close to Advaita Vedanta's distinction between Brahman and the many \emph{j\={\i}vas} under \emph{up\={a}dhis}, but reinterpreted in a Christological and value-laden key.

Biblical theology affirms this metaphysical structure with remarkable clarity. The doctrine of creation is not a claim of separation but of ongoing divine presence: ``In Him all things hold together'' (Colossians 1:17). God declares through Jeremiah, ``Do I not fill heaven and earth?'' (Jeremiah 23:24), establishing the Whole as the field in which all slices arise. Paul affirms that all creatures live within this divine milieu: ``For from Him and through Him and to Him are all things'' (Romans 11:36). The Logos theology of John identifies Christ as both origin and medium of all appearance: ``All things came into being through Him, and without Him not one thing came into being'' (John 1:3). And the psalmist expresses the intimacy of divine presence in each slice: ``Where can I go from Your Spirit?\ldots{} If I ascend to the heavens, You are there; if I make my bed in the depths, You are there also'' (Psalm 139:7--8). These passages ground the framework's assertion that slices are not separate entities but finite expressions of the One within whom they live, move, and have their being.

\section{Deep Patterning and Attention}

Beneath the layer of narrative memory there is a deeper level of organization that shapes how a slice interprets and responds to the world. This deeper layer is \emph{deep patterning}. It includes emotional imprints, learned reactions, attachment styles, internalized values, and implicit expectations of safety and meaning. It is not a set of discrete, recallable episodes, but a configuration of tendencies and sensitivities --- the weights of the slice, in analogy with machine learning systems.

Scripture presents this deep interior structure as the primary arena of spiritual formation. ``The purposes of a person's heart are deep waters'' (Proverbs 20:5), highlighting the depth and opacity of these dispositions. Jesus locates the moral and perceptual center in this layer: ``Out of the abundance of the heart the mouth speaks'' (Matthew 12:34) and ``from within, out of the heart, come evil thoughts'' (Mark 7:21). Paul names this domain the ``inner self'' (Romans 7:22) and contrasts its renewal with the corruption of the ``old self'' (Ephesians 4:22--23). The Psalmist prays for change specifically in this substrate: ``Create in me a clean heart, O God'' (Psalm 51:10). These passages confirm that deep patterning---our implicit habits of interpreting, desiring, and responding---is precisely the level at which biblical transformation occurs.

Attention is the dynamic movement of awareness that operates on top of this deep patterning. At any moment, attention determines what is foreground and what is background, which options are perceived as available, and what is experienced as meaningful or meaningless. Over time, patterns of attention can reshape deep patterning. Repeated attention to certain meanings, actions, or relational stances gradually reweights the inner structure of the slice.

In cognitive science, attention can be related to precision-weighting in predictive processing and active inference models. On these accounts, agents minimize prediction error by selectively sampling and weighting sensory inputs, revising their internal models, and reallocating attention to informative signals \citep{friston2010free}. In this view, a slice acts so as to avoid surprising states by constantly updating its generative model and moving attention to reduce important errors. This parallels the way deep patterning and attention interact in the present framework.

In classical Vedantic language, something like deep patterning corresponds to the subtle and causal bodies that store dispositions (\emph{sa\m sk\={a}ras}) and ignorance, while attention corresponds to the shifting identification of the \emph{j\={\i}va} with different layers of experience \citep{vivekanandajnana, sarvapriyananda2020advaita}. Practices of discriminative insight, meditation, and devotion aim to loosen unhealthy grooves in this patterning so that the ever-free Self can be recognized. In Wolfram's observer-theoretic terms, attention configures which microstates are treated as equivalent and which macroscopic features are extracted by an observer embedded in the underlying computational universe \citep{wolfram2021ruliad, wolfram2023observer}. Observers are finite structures that aggregate many underlying states into stable narratives. Within this framework, attention is also the primary locus of cooperation between the slice and the Whole: it is where the Whole's value gradient can be received or resisted.

The biblical tradition places exceptional emphasis on this attentional faculty as the site of cooperation with God. Jesus warns His disciples to ``watch and pray'' so that they do not fall into misalignment (Matthew 26:41). The Psalmist links attention with stability: ``I have set the Lord always before me; because He is at my right hand, I shall not be shaken'' (Psalm 16:8). Paul frames attention as an active discipline: ``Take every thought captive to obey Christ'' (2 Corinthians 10:5). Isaiah connects attentional steadfastness with inner peace: ``You keep him in perfect peace whose mind is stayed on You'' (Isaiah 26:3). Hebrews identifies attention as the means of spiritual perseverance: ``Fix your eyes on Jesus, the author and perfecter of our faith'' (Hebrews 12:2). In all these passages, attention is the living hinge between divine initiative and human response---the precise region where a slice either aligns with or resists the movement of the Whole.

\section{The Value Gradient and Evil}

The framework does not treat reality as morally neutral. The Whole exhibits a \emph{lean} or value gradient: a tendency to favor patterns that sustain coherence, life, mutuality, and self-giving love, and to disfavor patterns that collapse into fragmentation, cruelty, and meaning-destruction. This is not an external moral law imposed from outside the system. It is more akin to a selection pressure: certain configurations of deep patterning and attention are more stable, fruitful, and resonant with the Whole than others.

The biblical witness confirms this intrinsic moral orientation of creation. Scripture repeatedly describes reality as ordered toward righteousness, justice, and love: ``Righteousness and justice are the foundation of Your throne; steadfast love and faithfulness go before You'' (Psalm 89:14). Jesus teaches that the entire law aims at love of God and neighbor (Matthew 22:37--40), revealing a moral geometry built into creation. Paul describes creation as groaning for alignment: ``The creation waits in eager expectation for the children of God to be revealed\ldots{} creation itself will be liberated from its bondage to decay'' (Romans 8:19--21). Evil, in the biblical imagination, is not a rival substance but a distortion: ``All flesh had corrupted their way on the earth'' (Genesis 6:12). The prophets portray evil as fragmentation: ``Their feet rush into sin\ldots{} ruin and misery mark their ways, and the way of peace they do not know'' (Isaiah 59:7--8). These passages confirm that creation has a built-in moral gradient oriented toward coherence and love, and that evil is the anti-alignment that collapses coherence into fragmentation.

Within this gradient, evil is not a rival being or a second god. Evil is destructive misalignment and active anti-alignment with the Whole's lean. It appears as purposeless cruelty, the deliberate shattering of trust, the corrosion of meaning, the inversion of responsibility, and the cultivation of fragmentation for its own sake. This account is compatible with traditional notions of sin as a missing of the mark (hamartia), and with patristic talk of corruption and a slide toward non-being \citep{athanasius2011incarnation}, but it grounds this language in a structural account of coherence and fragmentation.

By construing evil as anti-alignment rather than as an independent substance, the framework avoids dualism while still taking seriously the real harm and suffering experienced within slices. The Whole does not endorse or require such patterns; rather, they represent failures of cooperation between slice-level aim and the underlying value gradient.

\section{Lives as Learning Trajectories}

Each finite life is interpreted as a learning trajectory of the Whole under that slice's constraints, much like a training run in machine learning. As experiences accumulate, deep patterning is updated: certain responses, interpretations, and relational possibilities become more or less likely. In this way, the Whole learns what it is like to be this configuration, in this history, under these pressures.

The biblical narrative consistently portrays human life as a divinely overseen trajectory of growth, formation, and transformation. The Psalmist declares that ``all the days ordained for me were written in Your book before one of them came to be'' (Psalm 139:16), affirming the meaningful structure of each life’s path. Paul likewise teaches that ``we are His workmanship'' (Ephesians 2:10), suggesting that God actively shapes each person’s unfolding. James frames suffering as formative: ``The testing of your faith produces perseverance\ldots{} that you may be mature and complete'' (James 1:3--4). Peter describes trials as refining fire (1 Peter 1:6--7). Jesus speaks of developmental transformation using agricultural imagery: ``By their fruits you will know them'' (Matthew 7:16). These passages ground the claim that each life unfolds as a learning trajectory through which the Whole---as God---shapes, tests, refines, and ultimately restores the deep patterning of the slice.

When a slice dissolves --- for example, at biological death --- the narrative ego and its explicit memories do not continue as a discrete, persisting object. The interface is gone. However, the deep patterning that developed during the life is not simply erased. It has become part of the Whole's overall configuration: the weights have been updated. In this sense, nothing lived is wasted. Alignments and misalignments both matter, because they leave marks on the structure of reality.

This picture differs from both simple annihilation and from naïve persistence of a fixed ego. It suggests that what is real and enduring is the contribution made to the Whole's patterning: aligned structures are retained and integrated, while fragmented patterns lose stability and decay toward noise. In dynamical and computational terms, each life traces a trajectory through the space of possible configurations under its rules and constraints, analogous to how learning under the free-energy principle adjusts an agent's generative model and how evolution explores a rulial ensemble of rules \citep{friston2010free, wolfram2021ruliad}. In process-theological and Vedantic terms, this is the ongoing creative advance and the accumulation of karmic patterns and dispositions.

Theologically, this way of speaking can sound uncomfortably close either to annihilationism, in which the person simply ceases to be, or to an impersonal absorption into the Whole. The Christian doctrine of resurrection, however, insists that creatures are restored by God as recognizably themselves before God and one another. In the terms of this framework, such continuity would mean that the Whole is able to re-instantiate a new, transfigured interface whose deep configuration is continuous with the deep patterning shaped in the learning trajectory, healed of fragmentation and decay. Eschatological judgment and mercy would not evaluate a detachable ego standing over against the Whole, but the pattern actually lived and given back to the Whole, which is then either intensified into stable alignment or allowed to collapse toward noise. The aim here is not to supply a full eschatology, but to signal that the learning-trajectory picture can coexist with a robust account of resurrection and personal recognition, provided that identity is understood in terms of enduring patterns of relation and alignment rather than numerical sameness of interface.

\section{The Christ Pattern and the Spirit}

Within this ontology, the \emph{Christ pattern} is the clearest revelation of the Whole's lean under human constraint. The Christ pattern is the Whole appearing as a slice in perfect alignment: self-giving love, truth without cruelty, mercy without enabling injustice, and coherence under maximal pressure, including suffering and death. In the Christian narrative, this pattern is embodied in the life, death, and resurrection of Jesus of Nazareth \citep{athanasius2011incarnation, wright1996jesus}.

Scripture identifies Jesus Christ as the definitive revelation of God’s nature under the constraints of human life. He is ``the image of the invisible God'' (Colossians 1:15), ``the exact representation of His being'' (Hebrews 1:3), and the Logos through whom all things were made (John 1:3). In Christ, the Whole discloses itself in finite form: ``In Him the whole fullness of deity dwells bodily'' (Colossians 2:9). The Gospels portray Jesus as perfectly aligned with the Father’s will: ``My food is to do the will of Him who sent Me'' (John 4:34). In Gethsemane He embodies alignment under maximal pressure: ``Not my will, but Yours be done'' (Luke 22:42). Paul presents Christ’s self-emptying as the archetype for all slices: ``He humbled Himself\ldots{} becoming obedient to the point of death'' (Philippians 2:8). The resurrection vindicates this alignment, revealing that the divine pattern of self-giving love is the deepest principle of reality: ``Death could not hold Him'' (Acts 2:24). These passages establish the Christ pattern as the Whole-as-slice without distortion---the perfect disclosure of divine nature under constraint.

The Passion, in particular, can be read as the stress test of alignment. Faced with betrayal, violence, and apparent abandonment, the Christ pattern does not collapse into retaliatory destruction or despair. Instead, it maintains self-giving love and trust in the Whole to the end. In the terms of this framework, the Christ pattern is the Whole-as-slice without distortion.

Crucially, this pattern is not an abstract archetype floating above the concrete figure of Christ. In classical Nicene theology, the one who lives this pattern is the eternal Word of the Father, through whom all things were made and in whom all things hold together (for example, John 1:1--3; Colossians 1:15--20; Hebrews 1:1--3). In Athanasius's account, the same Word who creates enters into the ``corruption'' brought by sin in order to abolish death and renew the image, so that creatures may share once more in incorruptible life \citep{athanasius2011incarnation}. The language of ``pattern'' here is descriptive: it names the form of life of this person under human constraint, rather than positing a separate principle alongside him. This guards against reducing Christ to a merely exemplary slice while preserving the metaphysical claim that the Whole itself is disclosed in him.

The Spirit, by contrast, is understood as the Whole's \emph{active attention} within and between slices. Where the Christ pattern is the form of perfect alignment under constraint, the Spirit is the movement that illumines, convicts, comforts, guides, and rewires deep patterning over time. In traditional Christian language, the Spirit is the indwelling presence of God who leads persons into truth, empowers acts of love, and builds up communities. In the present framework, this is precisely the activity of the Whole's lean within slice-level attention, and it is personal rather than impersonal: the Spirit speaks, grieves, intercedes, and binds persons together in love.

Scripture depicts the Holy Spirit precisely in these terms of active divine presence, illumination, and transformative guidance. Jesus promises that ``the Spirit of truth\ldots{} will guide you into all truth'' (John 16:13), emphasizing the Spirit's role as the agent of divine understanding. Paul teaches that ``the Spirit searches all things, even the deep things of God'' (1 Corinthians 2:10), indicating an interior, attentional movement within the divine life itself. The Spirit is the indwelling presence who forms alignment from within: ``Walk by the Spirit, and you will not gratify the desires of the flesh'' (Galatians 5:16). The same Spirit ``bears witness with our spirit that we are children of God'' (Romans 8:16), shaping identity at the level of deep patterning. The Spirit participates in the refinement of attention: ``We all\ldots{} beholding the glory of the Lord, are being transformed into the same image from glory to glory, as by the Spirit of the Lord'' (2 Corinthians 3:18). And the Spirit intercedes with intimate inwardness: ``The Spirit Himself intercedes for us with groanings too deep for words'' (Romans 8:26). These passages reveal the Spirit as the Whole's dynamic, relational self-gift---the divine attention moving within and between slices to restore alignment with the Christ pattern.

\section{A Trinitarian Mapping}

These reflections yield a structural mapping of the Christian doctrine of the Trinity. In classical trinitarian theology, God is one being in three persons: Father, Son, and Spirit. The unity is real (one God), and the distinctions are also real (three hypostases, not three masks). Historically, various models have been used to explicate this, including relational and psychological analogies \citep{augustine1991trinity, rahner1997trinity}. The present framework proposes the following correspondence:

\begin{itemize}
    \item The \textbf{Father} corresponds to the Whole-as-ground: undivided reality, source, and value gradient.
    \item The \textbf{Son} corresponds to the Whole-as-form: the Christ pattern, the Whole under finite constraint in perfect alignment, the image of the invisible Whole.
    \item The \textbf{Spirit} corresponds to the Whole-as-attention: the active presence, guidance, and transformative movement of the Whole within and between slices.
\end{itemize}

This structural mapping reflects the scriptural witness to God’s triune self-disclosure. The Father is repeatedly identified as the source and ground of all reality: ``From Him and through Him and to Him are all things'' (Romans 11:36) and ``one God and Father of all, who is over all and through all and in all'' (Ephesians 4:6). The Son is the visible and embodied self-expression of the Father: ``The image of the invisible God'' (Colossians 1:15), ``the exact imprint of His nature'' (Hebrews 1:3), and the Word through whom all things were made (John 1:3). The Spirit is the divine presence dynamically active within creation: ``The Spirit gives life'' (John 6:63), ``the Spirit searches all things'' (1 Corinthians 2:10), and ``the Spirit will guide you into all truth'' (John 16:13). These texts affirm that the triune God is one Whole who eternally exists as Source, Expression, and Dynamic Presence---precisely the structural pattern identified by this framework.

This mapping is not intended merely as metaphor. It is offered as a structural ontology: a way of understanding the threefold self-expression of the one Whole that resonates with, and is constrained by, Christian trinitarian commitments. It should not be read as simple modalism. Father, Son, and Spirit are not three successive roles that one agent sometimes plays, but three irreducible, mutually indwelling ways in which the Whole exists and relates.

Ontologically, the Whole is eternally Father, Son, and Spirit; epistemically, finite slices come to know this triune life only as they are drawn into relation with Christ in the Spirit.

The correspondence is therefore asymmetric. We are not defining the Trinity by the Whole, but rather using trinitarian revelation --- above all in the history of Jesus and the sending of the Spirit --- to say more precisely what ``Whole'' must mean. The economic missions of Son and Spirit, in which the Word becomes flesh and the Spirit indwells and transforms creatures, reveal how the Whole relates to slices, and thus disclose the inner life of the Whole itself, in harmony with the axiom that the economic Trinity is the immanent Trinity \citep{rahner1997trinity}.

\section{Religion, Aim, and Human Agency}

Within this framework, religious traditions are understood primarily as pointer systems: embedded, historically conditioned guidance structures for orienting finite attention toward alignment with the value gradient. Doctrinal claims, narrative traditions, ritual practices, and ethical norms all function --- at their best --- as tools for shaping deep patterning and attention. The goal is not conceptual accuracy for its own sake, but reconfiguration of lives toward coherence, life, mutuality, and self-giving love.

Scripture presents the life of faith in precisely these attentional and formational terms. Moses exhorts Israel to ``set your heart'' on the words of God (Deuteronomy 32:46), linking religious instruction to the deep patterning of the inner life. Jesus frames discipleship as the reconfiguration of perception and desire: ``Where your treasure is, there your heart will be also'' (Matthew 6:21). Paul describes religious life as the renewal of the mind: ``Be transformed by the renewing of your mind, so that you may discern the will of God'' (Romans 12:2). The early church is said to ``devote themselves'' to teaching, fellowship, and prayer (Acts 2:42), reflecting communal practices that shape attention and reinforce alignment. The book of Hebrews calls believers to mutual formation: ``Encourage one another daily\ldots{} so that none of you may be hardened by sin's deceitfulness'' (Hebrews 3:13). These passages reveal that religion, in the biblical witness, is a set of practices and perceptions ordered toward the alignment of human lives with the divine Whole.

This distinction foregrounds the difference between doctrinal structure and personal encounter. Ontologically, the Whole is eternally Father, Son, and Spirit; epistemically, finite slices come to know this triune life only as they are drawn into relation with Christ in the Spirit. The decisive interface is attention: the real, lived act of turning toward Jesus Christ. Doctrine can guide that turning, but it is the attention itself --- embodied, affective, surrendered --- that constitutes the actual cooperation. In this sense, religious traditions train perception. They do not cause alignment, but they can facilitate it.

This priority of encounter over mere conceptual correctness is a consistent theme of Scripture. Jesus calls disciples not simply to think about Him, but to abide in Him: ``Abide in me, and I in you'' (John 15:4). James insists that faith becomes real only when embodied: ``Be doers of the word, and not hearers only'' (James 1:22). The Psalms describe relational attention as the heart of spiritual life: ``Seek His face continually'' (Psalm 105:4). The prophets portray God's desire for a responsive people: ``Return to me with all your heart'' (Joel 2:12). Paul holds divine grace and human agency together: ``Work out your salvation\ldots{} for it is God who works in you'' (Philippians 2:12--13). These texts underscore the framework's claim: transformation happens not through mere intellectual assent but through relational, attentional participation in the divine life revealed in Christ and mediated by the Spirit.

Because attention participates in shaping deep patterning, and deep patterning in turn affects the shape of the Whole's experience through a slice, religious practice becomes a site of ontological significance. A spiritual tradition that repeatedly points attention toward reactive fear or tribal superiority reshapes the Whole's presence in that configuration. One that persistently trains attention toward surrender, self-giving, and inclusive mutuality reshapes it otherwise. This does not override grace or render alignment a merit-based outcome. It simply reflects the structural fact that what a slice attends to matters, and that attention can be formed.

Scripture presents this communal dimension of spiritual formation as essential rather than optional. The author of Hebrews commands believers to ``consider how to stir up one another to love and good works'' (Hebrews 10:24), emphasizing the way communities shape attention and desire. Paul teaches that the church is a body in which ``each part works properly'' to build the whole into maturity (Ephesians 4:16). Jesus describes the relational dynamics of alignment when He says, ``By this everyone will know that you are my disciples, if you love one another'' (John 13:35). Conversely, Scripture warns that malformed communities can produce collective misalignment: ``Bad company corrupts good character'' (1 Corinthians 15:33) and ``the blind lead the blind, and both fall into a pit'' (Matthew 15:14). These passages underscore that alignment is not merely an individual process but a relational and communal one: the patterns of the Whole are mediated not only within slices but also between them, as shared attention, practice, and love form the social environment in which deep patterning is shaped.

This gives a natural account of both the diversity of religions and their unequal effects. The Whole is not sectarian; it can show up wherever attention is available and alignment is possible. But not all systems equally facilitate that availability. Some traditions encode severe misalignments --- fueled by empire, fear, or control --- and thereby train spiritual constriction. Others, including many within Christianity, bear patterns of remarkable openness, courage, and self-giving love. Even these are imperfect, finite pointer systems. Their task is not to produce certainty but to guide cooperation.

Some slices appear to receive, by providence or context, an unusual degree of spiritual leverage --- a kind of directory-level write access. These are lives that, through suffering or surrender, seem capable of initiating real reconfiguration in the deep patterning of others. Religious traditions often name such lives saints, prophets, or bodhisattvas. Their significance is not in the quantity of followers or prestige, but in how their learning trajectories produce new alignments in the Whole under constraint. They are, in this framework, local intensifications of the Christ pattern.

Scripture affirms that certain individuals, by divine calling or disposition, bear a unique intensity of alignment that shapes the patterning of others. Paul teaches that grace is apportioned differently: ``To each one of us grace has been given as Christ apportioned it'' (Ephesians 4:7), and that some are given to the community as apostles, prophets, evangelists, pastors, and teachers ``to equip the saints for the work of ministry'' (Ephesians 4:11--12). Hebrews presents a ``great cloud of witnesses'' (Hebrews 12:1) whose lives exert ongoing formative influence. James declares that ``the prayer of a righteous person is powerful and effective'' (James 5:16), implying differential spiritual leverage. Jesus Himself teaches that those who embody His pattern become ``the light of the world'' (Matthew 5:14), radiating alignment into the lives of others. These passages confirm that Scripture recognizes certain lives as concentrated sites of divine influence---precisely what this framework describes as local intensifications of the Christ pattern.

This view does not require a strong notion of libertarian free will. Finite slices operate under constraint, shaped by prior patterning and limited scope. But within those constraints, attention still carries causal weight. It is not a ghostly override, but a structurally embedded capacity to cooperate or resist. Human agency is neither absolute nor illusory; it is real in its zone. That zone --- the moment of attention, the yes or no to what appears --- is small but potent.

Scripture presents human agency in precisely this bounded and cooperative way. God sets before His people real choices---``I have set before you life and death\ldots{} choose life'' (Deuteronomy 30:19)---yet Jesus reminds them that ``apart from Me you can do nothing'' (John 15:5). Paul holds divine sovereignty and human responsiveness together: ``Work out your salvation\ldots{} for it is God who works in you both to will and to work'' (Philippians 2:12--13). The exhortations of the New Testament assume genuine, though constrained, agency: ``Do not quench the Spirit'' (1 Thessalonians 5:19); ``Walk by the Spirit'' (Galatians 5:16); ``If you love Me, keep My commandments'' (John 14:15). These passages portray human agency as the real but limited capacity to cooperate with or resist the transformative movement of the Whole---the Spirit's work of aligning the slice with the Christ pattern.

The practical implications are twofold. First, spiritual formation becomes a kind of cooperative physics: learning to aim attention, reshape patterning, and align with the gradient of self-giving love. Second, religious communities can be evaluated not by orthodoxy alone but by the kinds of deep patterning they tend to produce. Do they cultivate freedom, compassion, and mutual participation? Or do they train division, fear, and reactive control? The test is not doctrinal fidelity but transformation. What the system points to --- and what kind of person it tends to make --- reveals whether it aligns with the value gradient or works against it.

\section{Comparative and Future Directions}

The ontology developed here overlaps in structure with several existing traditions while also modifying them. From process philosophy it takes the idea of a dynamic, internally related Whole and the primacy of process over substance \citep{whitehead1978process, whiteheadSEP}. The Whole in this framework is closer to Whitehead's creative advance than to a static Absolute; slices are akin to complexes of occasions with their own perspectives and inherited data. Whitehead's critique of ``simple location'' and of a materialism that treats matter as self-contained points supports a similar rejection of isolated substances in favor of relations within a wider field \citep{whitehead1978process, whiteheadSEP}. At the same time, the framework departs from classical process theism by locating the decisive self-disclosure of the Whole not in a generic God--world relation but in the concrete Christ-pattern and a trinitarian structure.

The scriptural witness itself provides the foundational metaphysical scaffolding that these traditions echo in diverse ways. The wisdom literature affirms a cosmos ordered by divine reason: ``The Lord by wisdom founded the earth; by understanding He established the heavens'' (Proverbs 3:19). John's Prologue identifies this ordering principle with the Logos, through whom all things were made (John 1:3). Paul describes creation as internally related to its source: ``All things are from Him and through Him and to Him'' (Romans 11:36). The relational ontology of the New Testament---``in Him all things hold together'' (Colossians 1:17)---mirrors the participatory structures emphasized by process thinkers and contemplative traditions alike. Even the biblical account of human transformation resonates with nondual and processual motifs: ``We are being transformed into His image from glory to glory'' (2 Corinthians 3:18). These passages demonstrate that Scripture itself anticipates many of the metaphysical intuitions developed in philosophical, contemplative, and computational traditions, while grounding them in a distinctively Christ-centered frame.

Process theologians such as Cobb and Viney extend Whitehead's ontology to a vision of God as the dynamic Whole in whom all finite experiences inhere. God is not remote but immanent, acting through persuasive love that lures creatures toward harmony and deeper value \citep{cobb2017process, viney2018processtheism}. The cosmos, in this view, exhibits a gradient of increasing coherence and intensity---resonating with the value gradient developed here. Evil is not a rival force but a failure of alignment: a breakdown in the cooperation between finite freedom and the Whole's aim for integration.

From predictive processing and active inference the framework takes the view that finite agents are constraint-laden prediction machines whose internal models and precisions are updated through experience \citep{friston2010free}. Deep patterning corresponds to the configuration of a slice's generative model, while attention modulates prediction-error weighting and sampling. Over time, learning trajectories reshape this patterning. Metzinger's self-model theory reinforces this picture, arguing that what we call ``self'' is not a substance but a virtual construct---a dynamic model with no fixed essence \citep{metzinger2003being}. Varela, Thompson, and Rosch similarly emphasize that identity is enacted through conditioned processes and historical interactions \citep{varela1991embodied}. These models support the framework's view of lives as open-ended learning arcs of the Whole under constraint.

From Advaita Ved\={a}nta it takes the insight that the apparent plurality of selves is Brahman manifest under limiting conditions. The classical doctrine that the \emph{j\={\i}va} is Brahman appearing under \emph{up\={a}dhis} (limiting adjuncts), shaped by \emph{sa\m sk\={a}ras}, mirrors the slice--constraint--pattern structure developed here \citep{vivekanandajnana, sarvapriyananda2020advaita, fowler1989advaita}. Liberation arises when attention is withdrawn from these patterns and reoriented toward the undivided Whole. The present framework affirms this structural picture while rejecting world-negation: it insists that created patterns matter eternally and that the Whole's nature is self-giving love rather than pure consciousness.

From patristic and mystical theology, particularly Gregory of Nyssa, Maximus the Confessor, and Meister Eckhart, the framework draws further support. Gregory's view that all creation exists within the divine mind aligns with the Whole as the ground of all slices \citep{schooping2015touching}. His notion of \emph{epektasis}---endless movement deeper into God---resonates with learning trajectories that never cease. Maximus's vision that the Logos is the inner principle of all things, and that creation is incarnation unfolding, directly parallels the Christ-pattern as the Whole in form \citep{wood2022whole}. Eckhart's teaching that the soul's ground is one with the Godhead and that the Trinity unfolds within the soul reflects a nondual participation in the trinitarian Whole \citep{griffioen2023eckhart}.

Teilhard de Chardin's evolutionary Christology adds a modern dimension to this lineage. He portrays the cosmos as a field converging toward the Omega Point---identified with the Cosmic Christ---and driven by love as its deepest force \citep{teilhard1959phenomenon}. Each conscious being, for Teilhard, is an awakening of the Whole gaining awareness of itself. This vision aligns with the framework's conception of self as the Whole appearing under constraint and love as the binding energy of alignment. Similarly, Sheldrake interprets the Trinity structurally as ground (Father), form (Son), and energy (Spirit), mirroring the mapping developed here \citep{sheldrake2019ways}.

Taken together, these sources converge on a shared structural pattern: a single dynamic Whole, many finite perspectives shaped by deep patterning, and a pull toward richer, more loving forms of coherence. Whether framed in Whitehead's divine aims, Friston's free-energy minimization, Vedantic nonduality, observer-based computation, or patristic Christology, each system describes a world in which attention and aim shape the real. The Whole-and-slice framework is offered as a Christ-patterned synthesis of these intuitions.

Future work might sharpen technical links to formal models, including integrated information theory and non-equilibrium thermodynamics. It could also extend the ethical and institutional consequences: what kinds of systems or technologies align with the value gradient? Finally, further theological reflection could clarify the implications for divine immutability, eschatology, and the transformation of deep patterning through practices of prayer, love, and discernment.

Biblical eschatology reinforces this trajectory-based, participatory, and transformative ontology. Paul envisions creation itself as moving toward liberation and renewal: ``The creation itself will be set free from its bondage to decay and brought into the freedom of the glory of the children of God'' (Romans 8:21). The resurrection of Christ is described as ``the firstfruits'' (1 Corinthians 15:20), revealing the pattern of renewed embodiment and healed identity that awaits all slices. The final telos is expressed in the declaration that ``God will be all in all'' (1 Corinthians 15:28), a vision of ultimate coherence in which every fragment of creaturely life is gathered, transfigured, and integrated into the divine Whole. Revelation portrays this as the consummation of divine presence: ``Behold, the dwelling of God is with humanity'' (Revelation 21:3). These passages confirm that the learning trajectories traced in finite lives, the Christ-pattern of perfect alignment, and the Spirit's ongoing reconfiguration of deep patterning all point toward a final state in which the Whole and its slices are united in consummate harmony.

\section{Conclusion}

This paper has developed a metaphysical and theological framework in which finite lives are understood as constraint-bounded appearances of a single undivided Whole. Each finite life, or slice, carries deep patterning that shapes what the Whole can experience through it. Attention modulates this patterning and, in doing so, becomes the primary site of cooperation or resistance. The Whole is structured by a value gradient toward coherence, life, and self-giving love. Evil is not a rival substance but a destructive misalignment or active anti-alignment with that gradient.

The Christ pattern shows what full alignment looks like under constraint. In Jesus Christ, the Whole lives a finite trajectory in perfect correspondence with its own value gradient, and thus reveals the form and aim of the Whole's action in the world. The Spirit continues this action by drawing other slices into cooperation. Trinitarian language --- understood structurally as Whole-as-ground (Father), Whole-as-form (Son), and Whole-as-attention (Spirit) --- becomes a map of how divine life moves through and within constraint.

This framework reframes religious life as attention-based cooperation. Traditions are pointer systems for guiding that cooperation. Lives are learning trajectories in which the Whole experiments with forms of self-giving under pressure. Identity is not defined by static ego or substance, but by enduring configurations of pattern and attention. Resurrection becomes the re-instantiation of healed patterning; judgment, a diagnosis of misalignment. Agency is real within its limits, and what we attend to matters.

Theologically, this model offers a way to integrate science, contemplative insight, and classical doctrine without collapsing into reduction or dualism. It treats love not as moral decoration but as ontological structure. And it insists that each life is not a private possession, but an experiment in the real: a site where the Whole can come to know and give itself again.

This proposal is necessarily incomplete. It leaves open important questions about the precise relation between this ontology and classical doctrines of divine immutability and simplicity, the formalization of alignment and evil in more technical terms, and the detailed shape of eschatological hope and personal identity beyond the sketch offered here. These are not afterthoughts but intended directions for future work, in which the Whole-and-slice framework can be tested, corrected, and refined in deeper conversation with both theological and scientific interlocutors.

The biblical witness affirms this vision of reality as a participatory, Christ-shaped unfolding grounded in divine love. Paul declares that ``Christ in you'' is ``the hope of glory'' (Colossians 1:27), uniting the Whole and the slice in a single transformative mystery. John promises that ``when He appears, we shall be like Him, for we shall see Him as He is'' (1 John 3:2), describing the final healing and completion of deep patterning. Jesus teaches that the Spirit ``will be with you and in you'' (John 14:17), grounding the Spirit's attentional work as the ongoing reconfiguration of the person. And the entire biblical narrative culminates in the proclamation that ``the dwelling of God is with humanity'' (Revelation 21:3) and ``God will be all in all'' (1 Corinthians 15:28). These passages confirm that the Whole-and-slice ontology is not only philosophically coherent but scripturally faithful: a vision of reality in which God's triune life moves within creation to bring every slice into alignment with the fullness of divine love.

\bibliographystyle{apalike}
\bibliography{refs}

\end{document}
